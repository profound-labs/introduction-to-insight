\emph{Time and Place}

Focusing the mind on the body can be readily accomplished while sitting.
You need to find a time and a place which affords you calm and freedom
from disturbance.

A quiet room with not much in it to distract the mind is ideal; a
setting with light and space has a brightening and clearing effect,
while a cluttered and gloomy room has just the opposite. Timing is also
important, particularly as most people's days are quite structured with
routines. It is not especially productive to meditate when you have
something else to do, or when you're pressed for time. It's better to
set aside a period -- say, in the early morning or in the evening after
work -- when you can really give your full attention to the practice.
Begin with fifteen minutes or so. Practice sincerely with the
limitations of time and available energy, and avoid becoming mechanical
about the routine. Meditation practice, supported by genuine willingness
to investigate and make peace with oneself, will develop naturally in
terms of duration and skill.

\emph{Awareness of the Body}

The development of calm is aided by stability, and by a steady but
peaceful effort. If you can't feel settled, there's no peacefulness; if
there's no sense of application, you tend to daydream. One of the most
effective postures for the cultivation of the proper combination of
stillness and energy is sitting.

Use a posture that will keep your back straight without strain. A simple
upright chair may be helpful, or you may be able to use one of the lotus
postures (Illustrations and notes on posture are given later.) These
look awkward at first, but in time they provide a unique balance of
gentle firmness that gladdens the mind without tiring the body.

If the chin is tilted very slightly down this will help, but do not
allow the head to loll forward as this encourages drowsiness. Place the
hands on your lap, palms upward, one resting gently on the other with
the thumb -- tips touching. Take your time, and get the right balance.

Now, collect your attention, and begin to move it slowly down your body.
Notice the sensations. Relax any tensions, particularly in the face,
neck, and hands. Allow the eyelids to close or half close.

Investigate how you are feeling. Expectant or tense? Then relax your
attention a little. With this, the mind will probably calm down, and you
may find some thoughts drifting in reflections, daydreams, memories, or
doubts about whether you are doing it right! Instead of following or
contending with these thought patterns, bring more attention to the
body, which is a useful anchor for a wandering mind.

Cultivate a spirit of inquiry in your meditation attitude. Take your
time. Move your attention, for example, systematically from the crown of
the head down over the whole body. Notice the different sensations -
such as warmth, pulsing, numbness, and sensitivity -- in the joints of
each finger, the moisture of the palms, and the pulse in the wrist. Even
areas that may have no particular sensation, such as the forearms or the
earlobes, can be `swept over' in an attentive way. Notice how even the
lack of sensation is something the mind can be aware of. The constant
and sustained investigation is called mindfulness (sati) and is one of
the primary tools of Insight Meditation.

\newpage

\emph{Mindfulness of Breathing (Ānāpānasati)}

Instead of `body sweeping,' or after a preliminary period of this
practice, mindfulness can be developed through attention on the breath.

First, follow the sensation of your ordinary breath as it flows in
through the nostrils and fills the chest and abdomen. Then try
maintaining your attention at one point, either at the diaphragm or -- a
more refined location -- at the nostrils. Breath has a tranquilizing
quality, steady and relaxing if you don't force it; this is helped by an
upright posture. Your mind may wander, but keep patiently returning to
the breath.

It is not necessary to develop concentration to the point of excluding
everything else except the breath. Rather than to create a trance, the
purpose is to allow you to notice the workings of the mind, and to bring
a measure of peaceful clarity into it. The entire process -- gathering
your attention, noticing the breath, noticing that the mind has
wandered, and re-establishing your attention -- develops mindfulness,
patience, and insightful understanding. So don't be put off by apparent
`failure' -- simply begin again. Continuing in this way allows the mind
to eventually calm down.

If you get very restless or agitated, just relax. Practice being at
peace with yourself, listening to -- without necessarily believing in -
the voices of the mind.

If you feel drowsy, then put more care and attention into your body and
posture. Refining your attention or pursuing tranquillity at such times
will only make matters worse!

