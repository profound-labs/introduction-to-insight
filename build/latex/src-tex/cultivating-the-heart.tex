
Cultivating goodwill (\textit{mettā}) gives another dimension to the practice of
Insight. Meditation naturally teaches patience and tolerance, or at
least it shows the importance of these qualities. So you may well wish
to develop a more friendly and caring attitude towards yourself and
other people. In meditation, you can cultivate goodwill very
realistically.

%\vspace*{0.5\onelineskip}
Focus attention on the breath, which you will now be using as the means
of spreading kindness and goodwill. Begin with yourself, with your body.
Visualize the breath as a light, or see your awareness as being a warm
ray, and gradually sweep it over your body. Lightly focus your attention
on the centre of the chest, around the heart region. As you breathe in,
direct patient kindness towards yourself, perhaps with the thought, `May
I be well,' or `Peace.' As you breathe out, let the mood of that
thought, or the awareness of light, spread outward from the heart,
through the body, through the mind, and beyond yourself. `May others be
well.'

%\vspace*{0.5\onelineskip}
If you are experiencing negative states of mind, breathe in the
qualities of tolerance and forgiveness. Visualising the breath as having
a healing colour may be helpful. On the out-breath, let go -- of any
stress, worry or negativity -- and extend the sense of release through
the body, the mind, and beyond, as before.

This practice can form all or part of a period of meditation -- you have
to judge for yourself what is appropriate. The calming effect of
meditating with a kindly attitude is good for beginning a sitting, but
there will no doubt be times to use this approach for long periods, to
go deeply into the heart.

Always begin with what you are aware of, even if it seems trivial or
confused. Let your mind rest calmly on that -- whether it's boredom, an
aching knee, or the frustration of not feeling particularly kindly.
Allow these to be; practice being at peace with them. Recognise and
gently put aside any tendencies towards laziness, doubt or guilt.

Peacefulness can develop into a very nourishing kindness towards
yourself, if you first of all fully accept the presence of what you
dislike. Keep the attention steady, and open the heart to whatever you
experience. This does not imply approval of negative states, but allows
them a space wherein they can come and go.

Generating goodwill toward the world beyond yourself follows much the
same pattern. A simple way to spread kindness is to work in stages.\linebreak
Start with yourself, joining the sense of loving acceptance to the
movement of the breath. `May I be well.' Then, reflect on people you
love and respect, and wish them well, one by one. Move on to friendly
acquaintances, then to those towards whom you feel indifferent. `May
they be well.' Finally, bring to mind those people you fear or dislike,
and continue to send out wishes of goodwill.

This meditation can expand, in a movement of compassion, to include all
people in the world, in their many circumstances. And remember, you
don't have to feel that you love everyone in order to wish them well!

Kindness and compassion originate from the same source of goodwill, and
they broaden the mind beyond the purely personal perspective. If you're
not always trying to make things go the way you want them to; if you're
more accepting and receptive to yourself and others as they are,
compassion arises by itself. Compassion is the natural sensitivity of
the heart.

