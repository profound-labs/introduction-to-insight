
With the practice of Insight Meditation you will see your attitudes more
clearly, and come to know which are helpful and which create
difficulties. An open attitude can make even unpleasant experiences
insightful -- for instance, undertaking the way that the mind reacts
against pain and sickness. When you approach such experiences in this
way, you can often unwind the stress and resistance to pain, and
alleviate it to a great degree. On the other hand, an impatient streak
will have different results: becoming annoyed with others if they
disturb your meditation; being disappointed if your practice doesn't
seem to be progressing fast enough; falling into unpleasant moods over
insignificant matters. Meditation teaches us that peace of mind -- or
its absence -- essentially depends on whether or not we contemplate the
events of life in a spirit of reflection and open-mindedness.

By looking into your intentions and attitudes in the quiet of
meditation, you can investigate the relationship between desire and
dissatisfaction. See the causes of discontent: wanting what you don't
have; rejecting what you dislike; being unable to keep what you want.
This is especially oppressive when the subject of the discontent and
desire is yourself. No-one finds it easy to be at peace with personal
weakness, especially when so much social emphasis is placed on feeling
good, getting\linebreak ahead and having the best. Such expectations indeed make
it difficult to accept oneself as one is.

However, with the practice of Insight Meditation you discover a space in
which to stand back a little from what you think you are, from what you
think you have. Contemplating these perceptions, it becomes clearer that
you don't have any thing as `me' or `mine;' there are simply
experiences, which come and go through the mind. So if, for example,
you're looking into an irritating habit, rather than becoming depressed
by it, you don't reinforce it and the habit passes away. It may come
back again, but this time it's weaker, and you know what to do. Through
cultivating peaceful attention, mental content calms down and may even
fade out, leaving the mind clear and refreshed. Such is the ongoing path
of insight.

To be able to go to a still centre of awareness within the changing flow
of daily life is the sign of a mature practice, for insight deepens
immeasurably when it is able to spread to all experience. Try to use the
perspective of insight no matter what you are doing -- routine housework,
driving the car, having a cup of tea. Collect the awareness, rest it
steadily on what you are doing, and rouse a sense of inquiry into the
nature of the mind in the midst of activity. Using the practice to
centre on physical sensations, mental states, or eye, ear or nose
consciousness can develop an ongoing contemplation that turns mundane
tasks into foundations for insight.

Centred more and more in awareness, the mind becomes free to respond
skilfully to the moment, and there is greater harmony in life. This is
the way that meditation does `social work' -- by bringing awareness into
your life, it brings peace into the world. When you can abide peacefully
with the great variety of feelings that arise in consciousness, you are
able to live more openly with the world, and with yourself as you are.

