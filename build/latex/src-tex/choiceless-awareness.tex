
Meditation can also proceed without a meditation object, in a state of
pure contemplation, or `choiceless awareness.'

After calming the mind by one of the methods described above,
consciously put aside the meditation object. Observe the flow of mental
images and sensations just as they arise, without engaging in criticism
or praise. Notice any aversion and fascination; contemplate any
uncertainty, happiness, restlessness or tranquillity as it arises. You
can return to a meditation object (such as the breath) whenever the
sense of clarity diminishes, or if you begin to feel overwhelmed by
impressions. When a sense of steadiness returns, you can relinquish the
object again.

This practice of `bare attention' is well-suited for contemplating the
mental process. Along with observing the mind's particular
`ingredients,' we can turn our attention to the nature of the container.
As for the contents of the mind, Buddhist teaching points especially to
three simple, fundamental characteristics.

First, there is changeability (\emph{anicca}) -- the ceaseless beginning
and ending all things go through, the constant movement of the content
of the mind. This mind-stuff may be pleasant or unpleasant, but it is
never at rest.

There is also a persistent, often subtle, sense of dissatisfaction
(\emph{dukkha}). Unpleasant sensations easily evoke that sense, but even
a lovely experience creates a tug in the heart when it ends. So at the
best of moments there is still an inconclusive quality in what the mind
experiences, a somewhat unsatisfied feeling.

As the constant arising and passing of experiences and moods become
familiar, it also becomes clear that -- since there is no permanence in
them -- none of them really belong to you. And, when this mind-stuff is
silent -- revealing a bright spaciousness of mind -- there are no purely
personal characteristics to be found! This can be difficult to
comprehend, but in reality there is no `me' and no `mine' -- the
characteristic of `no-self,' or impersonality (\emph{anattā}).

Investigate fully and notice how these qualities pertain to all things,
physical and mental. No matter if your experiences are joyful or barely
endurable, this contemplation will lead to a calm and balanced
perspective on your life.

