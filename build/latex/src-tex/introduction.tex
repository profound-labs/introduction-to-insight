
The aim of this booklet is to serve as an introduction to the practice\linebreak
of Insight Meditation as taught within the tradition of Theravāda
Buddhism. You need not be familiar with the teachings of the Buddha to
make use of it, although such knowledge can help to clarify any personal
understanding you may develop through meditation.

The purpose of Insight Meditation is not to create a system of beliefs, but rather to
give guidance on how to see clearly into the nature of the mind. In
this way one gains firsthand understanding of the way things are,
without reliance on opinions or theories, a direct experience, which has
its own vitality. It also gives rise to the sense of deep calm that
comes from knowing something for oneself, beyond any doubt.

Insight Meditation is a key factor in the path that the Buddha offered for the
welfare of human beings; the only criterion is that one has to put it
into practice! These pages, therefore, describe a series of meditation
exercises, and practical advice on how to use them. It works best if the
reader follows the guide progressively, giving each sequence of
instructions a good workout before proceeding further.

The term `Insight Meditation' (\emph{samatha-vipassanā})\footnote{Knowledge of terms in Pali -- the canonical language of
Theravāda Buddhism -- is not necessary to begin the practice of
meditation. It can be useful, however, to provide reference points to
the large source of guidance in the Theravāda Canon, as well as to the
teaching of many contemporary masters who still find such words more
precise than their English equivalents.}
refers to
practices for the mind that develop calm (\emph{samatha}) through
sustained attention, and insight (\emph{vipassanā}) through reflection.
A fundamental technique for sustaining attention is focusing awareness
on the body; traditionally, this is practiced while sitting or walking.
The guide begins with some advice on this.

Reflection occurs quite naturally afterwards, when one is `comfortable' within the context of
the meditation exercise. There will be a sense of ease and interest, and
one begins to look around and become acquainted with the mind that is
meditating. This `looking around' is called contemplation, a personal
and direct seeing that can only be suggested by any technique. A few
ideas and guidance on this come in a later section.

