
As our insight deepens, we see more clearly the results of our actions -
the peace that good intention, sincerity and clear-mindedness promote,
and the trouble that confusion and carelessness create. It is this
greater sensitivity, observing in particular the distress we cause
ourselves and others, that often inspires us to want to live more
wisely. For true peace of mind, it is indispensable that formal
meditation be combined with a commitment to responsibility, and with
care for oneself and others.

There is really nothing mysterious about the path of Insight. In the
words of the Buddha, the way is simple: `Do good, refrain from doing
evil, and purify the mind.' It is a long-observed tradition, then, for
people who engage in spiritual practice to place great importance on
proper conduct. Many meditators undertake realistic moral vows -- such as
refraining from harming living beings, from careless use of sexuality,
from using intoxicants (alcohol and drugs), and from gossip and other
graceless speech habits -- to help their own inner clarity, and perhaps
gently encourage that of others.
