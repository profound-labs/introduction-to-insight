
These meditation exercises all serve to establish awareness of things as
they are. By bringing your mind fully onto experiences, you will notice
more clearly the state of the mind itself -- for example, whether you are
being lazy or over eager in your practice. With a little honest
appraisal, it becomes evident that the quality of the meditation
practice depends, not on the exercise being used, but on what you are
putting into it. Reflecting in this way, you will gain deeper insight
into your personality and habits.

There are some useful points to bear in mind whenever you meditate.
Consider whether you are beginning afresh each time -- or even better,
with each breath or footstep. If you don't practice with an open mind,
you may find yourself trying to recreate a past insight, or unwilling to
learn from your mistakes. Is there the right balance of energy whereby
you are doing all that you can without being over forceful? Are you
keeping in touch with what is actually happening in your mind, or using
a technique in a dull, mechanical way? As for concentration, it's good
to check whether you are putting aside concerns that are not immediate,
or letting yourself meander in thoughts and moods. Or, are you trying to
repress feelings without acknowledging them and responding wisely?

Proper concentration is that which unifies the heart and mind.
Reflecting in this way encourages you to develop a skilful approach. And
of course, reflection will show you more than how to meditate: it will
give you the clarity to understand yourself.

Remember, until you've developed some skill and ease with meditation,
it's best to use a meditation object, such as the breath, as a focus for
awareness and as an antidote for the overwhelming nature of the mind's
distractions. Even so, whatever your length of experience with the
practice, it is always helpful to return to awareness of the breath or
body. Developing this ability to begin again leads to stability and
ease. With a balanced practice, you realize more and more the way the
body and mind are, and see how to live with greater freedom and harmony.
This is the purpose and the fruit of Insight Meditation.

