
The ideal is an upright, alert posture. Slumping only increases the
pressure on the legs and discomfort in the back. It is important to
attend to your posture with wisdom, not insensitive will-power! Posture
will improve with time, but you need to work with the body, not use
force \emph{against} it.

Check your posture:
\begin{packeditemize}
\item
  Are the hips leaning back? This will cause a slump.
\item
  The small of the back should have its natural, unforced curve so that
  the abdomen is forward and `open.'
\item
  Imagine that someone is gently pushing between the shoulder blades,
  while keeping the muscles relaxed. This will give you an idea of
  whether you unconsciously `hunch' your shoulders (and hence close your
  chest).
\item
  Note, and gently release, any tension in the neck/shoulder region.
\end{packeditemize}

\newpage

If your posture feels tense or slack:
\begin{packeditemize}
\item
  Allow the spine to straighten by imagining the crown of the head as
  suspended from above. This also lets the chin tuck in slightly.
\item
  Keep the arms light and held back against the abdomen. If they are
  forward, they pull you out of balance.
\item
  Use a small firm cushion underneath and toward the back of the
  buttocks to support the angle of the hips.
\end{packeditemize}

For the legs:
\begin{packeditemize}
\item
  Practice some stretching exercises (like touching the toes with both
  legs stretched out, while sitting).
\item
  If you have a lot of pain during a period of sitting, change posture,
  sit on a small stool or chair, or stand up for awhile.
\item
  If you usually (or wish to) sit on or near the floor, experiment with
  cushions of different size and firmness, or try out one of the special
  meditation stools that are available.
\end{packeditemize}

\newpage

For drowsiness:
\begin{packeditemize}
\item
  Try meditating with your eyes open.
\item
  `Sweep' your attention systematically around the body.
\item
  Focus on the whole body and on physical sensations, rather than on a
  subtle object like the breath.
\item
  Stand up and walk mindfully for awhile in the open air.
\end{packeditemize}

For tension or headaches:
\begin{packeditemize}
\item
  You may be trying too hard -- this is not unusual! -- so lighten your
  concentration. For instance, you might move your attention to the
  sensation of the breath at the abdomen.
\item
  Generate the energy of goodwill (see `Cultivating the Heart') and
  direct it towards the area of tension.
\item
  Visualizing and spreading light through the body can be helpful in
  alleviating its aches and pains. Try actually focusing a benevolent
  light on an area of difficulty!
\end{packeditemize}

