
Many meditation exercises, such as the above `mindfulness of breathing,'
are practiced while sitting. However, walking is commonly alter\-nated
with sitting as a form for meditation. Apart from giving you dif\-ferent
things to notice, it's a skilful way to energise the practice if the
calming effect of sitting is making you dull.

If you have access to some open land, measure off about 25-30 paces'
length of level ground (or a clearly defined pathway between two trees),
as your meditation path. Stand at one end of the path, and compose your
mind on the sensations of the body. First, let the attention rest on the
feeling of the body standing upright, with the arms hanging naturally
and the hands lightly clasped in front or behind. Allow the eyes to gaze
at a point about three meters (10 feet) in front of you at ground
level, thus avoiding visual distraction. Now, walk gently, at a
deliberate but `normal' pace, to the end of the path. Stop. Focus on the
body standing for the period of a couple of breaths. Turn, and walk back
again. While walking, be aware of the general flow of physical
sensations, or more closely direct your attention to the feet. The
exercise for the mind is to keep bringing the attention back to the
sensation of the feet touching the ground, the spaces between each step,
and the feelings of stopping and starting.

Of course, the mind will wander. So it is important to cultivate
patience, and the resolve to begin again. Adjust the pace to suit your
state of mind -- vigorous when drowsy or trapped in obsessive thought,
firm but gentle when restless and impatient. At the end of the path,
stop; breathe in and out; `let go' of any restlessness, worry, calm,
bliss, memories or opinions about yourself. The `inner chatter' may stop
momentarily, or fade out. Begin again. In this way you continually
refresh the mind, and allow it to settle at its own rate.

\vspace*{-0.5\onelineskip}
In more confined spaces, alter the length of the path to suit what is
available. Alternatively, you can circumambulate a room, pausing after
each circumambulation for a few moments of standing. The period of
standing can be extended to several minutes, using `body sweeping.'

\vspace*{-0.5\onelineskip}
Walking brings energy and fluidity into the practice, so keep your pace
steady and just let changing conditions pass through the mind. Rather
than expecting the mind to be as still as it might be while sitting,
contemplate the flow of phenomena. It is remarkable how many times we
can become engrossed in a train of thought -- arriving at the end of the
path and `coming to' with a start -- but is natural for our untrained
minds to become absorbed in thoughts and moods. So instead of giving in
to impatience, learn how to let go, and begin again. A sense of ease and
calm may then arise, allowing the mind to become open and clear in a
natural, unforced way.

